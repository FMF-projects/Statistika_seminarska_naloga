\documentclass[A4paper, 11pt]{article}
\usepackage[slovene]{babel}
\usepackage[utf8]{inputenc}
\usepackage{array}
\usepackage{amsfonts}
\usepackage{amsmath}
\usepackage{pifont}
\usepackage{theorem}
\usepackage{authblk}
\usepackage{url}
\usepackage[normalem]{ulem}

\usepackage{graphicx}
\usepackage{subcaption}
\graphicspath{ {slike/} }

\title{Seminarska naloga iz statistike}
\author{Klementina Pirc}
\affil{Fakulteta za matematiko in fiziko \\ Oddelek za matematiko}
\date{julij 2020}

\newtheorem{definicija}{Definicija}
\theorembodyfont{\mdseries}
\newtheorem{zgled}{Zgled}

\newcommand{\cmark}{\ding{51}}
\newcommand{\xmark}{\ding{55}}

\begin{document}

\begin{titlepage} 

\maketitle
\thispagestyle{empty}
	
\end{titlepage}

Podrobnejša navodila nalog se nahajajo v datoteki \textit{27\_sem\_nal\_Klementina\_Pirc.pdf}

% 1. NALOGA

\section*{1. naloga}

V datoteki \textit{Kibergrad.csv} so podane informacije o 43 886 družinah. Spodnje naloge se navezujejo na stopnjo izobrazbe, ki jo imajo vodje gospodinjstev. Stopnje so označene s števili od 31 do 46. Naloge sem rešila s pomočjo programa Python in ustreznih knjižnic, postopek pa se nahaja v datoteki \textit{Kibergrad.py}

\subsection*{a)}
Na podlagi enostavnega slučajnega vzorca 200 družin poiščemo oceno za delež družin v celotni populaciji, katerih vodja gospodinjstva nima srednješolske izobrazbe. S pomočjo opisov stopenj izobrazbe ugotovimo, da moramo prešteti družine iz vzorca s stopnjo $<= 38$. Naj bo $d_i$ stopnja izobrazbe za i-to družino, definiramo 
\[x_i = \left \{
	\begin{array}{ll}
		1  & \mbox{če } d_i \leq 38 \\
		0 & \mbox{sicer}
	\end{array}
	\right.
\]
Sedaj lahko vzorčni delež $s$ izračunamo s formulo
\[ s = \frac{1}{n} \sum_{i=1}^{n} x_i \]
kjer je $n$ velikost vzorca, torej 200. Dobimo rezultat $s = 0.23$.  

\subsection*{b)}
Ocenimo standardno napako $se(s)$ in določimo 95\% interval zaupanja. Za izračun standardne napake moramo izračunati varianco $s$. Ker smo vzeli enostavni slučajno vzorec velja
\[ var(s) = \frac{1}{n} \frac{N - n}{N - 1} \tilde{\sigma}^2 \]
kjer je $N=43886$ velikost populacije in $\tilde{\sigma}^2$ nepristranska cenilka za populacijsko varianco. Torej
\[ 
\begin{split}
\tilde{\sigma}^2 = \frac{N - 1}{N (n - 1)} \sum_{i=1}^{n} (x_i - s)^2 \\
\Rightarrow var(s) & = \frac{N - n}{n N (n - 1)} \sum_{i=1}^{n} (x_i - s)^2 \\
		         & = \frac{N - n}{n N (n - 1)} \sum_{i=1}^{n} (x_i^2 - 2x_i s + s^2) \\
		         & = \frac{N - n}{n N (n - 1)} \sum_{i=1}^{n} (x_i - 2x_i s + s^2) \\
		         & = \frac{N - n}{N (n - 1)} (s - s^2)
\end{split}
\]
in zato
\[ se(s) = \sqrt{ \frac{N - n}{N (n - 1)} s (1 - s)} \]
Upoštevali smo $x_i ^2 = x_i$, kar velja, ker $x_i \in \{0,1\}$. Standardna napaka za izbrani vzorec je $0.029763$. \\
\\
Interval zaupanja dobimo po formuli $s \mp z_\alpha se(s)$. Ker želimo 95\% natančnost, je $z_\alpha = 1.96$ in tako dobimo interval $[0.171662, 0.288337]$.

\subsection*{c)}
Izračunamo populacijski delež $S$ in standardno napako $se(S)$ s formulama
\[ S = \frac{1}{N} \sum_{i=1}^{N} x_i  \qquad se(S) = \sqrt{\frac{\sigma ^2}{N}} = \sqrt{\frac{1}{N^2} \sum_{i=1}^{N} (x_i - S)^2} \]
ter dobimo $S=0.211502$ in $se(S)=0.001949$. Interval zaupanja torej pokrije populacijski delež, saj $S \in [0.171662, 0.288337]$. \\
\\
Razlika med vzorčnim in populacijskim deležem znaša $0.018497$, razlika med vzorčno in populacijsko standardno napako pa $0.027814$.

\subsection*{d)}
Izberemo še 99 vzorcev, določimo 95\% intervale zaupanja, ter jih narišemo skupaj z intervalom za prvi vzorec. S horizontalno črto označimo vrednost S. Vidimo, da populacijski delež pokrije 96 intervalov.

\includegraphics[scale=0.8]{Figure_1}

\subsection*{e)}
Izračunamo standardni odklon vzorčnih deležev  iz 100 prej izbranih vzorcev. Uporabimo formulo
\[ \sigma_s = \sqrt{\frac{1}{m} \sum_{i=1}{m} (s_i - \bar{s})^2} \] 
kjer je $m=100$ število vzorcev, $s_i$ vzorčni delež i-tega vzorca in $\bar{s}$ povprečje vzorčnih deležev. 
$\sigma_s = 0.028618$, razlika med populacijsko standardno napako $se(S)$ in $\sigma_s$ pa je 0.026668. 
% za vzorec velikosti 200?

\newpage

\subsection*{f)}
Sedaj izberemo 100 vzorcev velikosti 800 in ponovimo postopke iz točk d) in e).
\begin{itemize}
\item $\sigma_s = 0.013002$
\item $| \sigma_s - se(S) | = 0.011053$
\item populacijski delež S pokrije 95 intervalov zaupanja
\end{itemize}
% teli intervali so mal shady
% obrazlozitev in primerjava

\includegraphics[scale=0.8]{Figure_2}


% 2. NALOGA

\section*{2. naloga}

Želimo oceniti skupno vrednost inventarja z N enotami, kjer ima vsaka knjigovodsko vrednost $X$ in dejansko vrednost $Y$. Poznamo $X$ za vse enote in želimo oceniti vsoto dejanskih vrednosti vseh enot $\tau_Y$.
\[ \tau_Y = \sum_{i=1}^{N} Y_i \]
Vzamemo enostavni slučajni vzorec z $n$ enotami in predlagamo 3 možne cenilke

\[ \hat{\tau}_Y ^ {(0)} := N  \bar{Y} \qquad \hat{\tau}_Y ^ {(1)} := \tau_X + N ( \bar{Y} - \bar{X}) \qquad \hat{\tau}_Y ^ {(2)} := \frac{\bar{Y}}{\bar{X}} \tau_X \]


\subsection*{a)}
Dokažimo, da je $\hat{\tau_Y} ^ {(1)}$ nepristranska: $\dashuline{ E(\hat{\tau}_Y ^ {(1)}) = \tau_Y}$ in izračunajmo njeno varianco. Uporabili bomo naslednje oznake. \\
\\
$\tau_X$, $\tau_Y$ \ldots vsota vrednosti spremenljivke $X$ oz. $Y$ na celotnem inventarju, \\
$\mu_X$, $\mu_Y$ \ldots povprečna vrednost $X$ oz. $Y$ na celotnem inventarju, \\
$\sigma_X$, $\sigma_Y$ \ldots standardni odklon $X$ oz. $Y$ na vseh enotah inventarja, \\
$\rho_{XY}$ \ldots korelacijski koeficient med $X$ in $Y$ na celotnem inventarju, \\
$\bar{X}$, $\bar{Y}$ \ldots vzorčno povprečje $X$ oz. $Y$.


\[ 
\begin{split}
E(\hat{\tau}_Y ^ {(1)}) & = E(\tau_X + N (\bar{Y} - \bar{X})) \\
			       & = \tau_X  + N (E(\bar{Y}) - E(\bar{X})) \\
			       %& = \tau_X + N  \left ( \frac{1}{n} \sum_{i=1}^{n} ( E(Y_i) - E(X_i) ) \right ) \\ 
			       %& = \tau_X + N \left ( \frac{1}{n} \sum_{i=1}^{n} (\mu_Y - \mu_X) \right ) \\
			       & = \tau_X + N \mu_Y - N \mu_X \\
			       & = \tau_X + \tau_Y - \tau_X \\
			       & = \tau_Y
\end{split}
\]

\[
\begin{split}
var(\hat{\tau}_Y ^ {(1)})  & = var(\tau_X + N (\bar{Y} - \bar{X})) \\
				& = N^2 var( \bar{Y} - \bar{X} ) \\
				& = N^2 var( \bar{Y} + \bar{X} - 2 cov(\bar{Y},\bar{X}) ) \\
				& = N^2 ( \frac{\sigma_Y^2}{n} \frac{N - n}{N - 1} + \frac{\sigma_X^2}{n} \frac{N - n}{N - 1} - 2\rho_{XY} ) \\
				&= \frac{N^2}{n} \frac{N - n}{N - 1} (\sigma_Y^2 + \sigma_X^2) - 2 N^2 \rho_{XY}
\end{split}
\]


\subsection*{b)}
Privzamemo, da sta $X$ in $Y - X$ na populacijii neodvisni. Zanima nas, kdaj ima cenilka $\hat{\tau}_Y ^ {(1)}$ nižjo varianco kot cenilka $\hat{\tau}_Y ^ {(0)}$.

\[
\begin{split}
var(\hat{\tau}_Y ^ {(0)}) & = var(N \bar{Y}) \\
                                          & = \frac{N^2}{n} \frac{N - n}{N - 1} \sigma_Y^2
\end{split}
\]
\\
\[
\begin{split}
var(\hat{\tau}_Y ^ {(1)}) & < var(\hat{\tau}_Y ^ {(0)}) \\
 \frac{N^2}{n} \frac{N - n}{N - 1} (\sigma_Y^2 + \sigma_X^2) - 2 N^2 \rho_{XY} & < \frac{N^2}{n} \frac{N - n}{N - 1} \sigma_Y^2 \\
 \frac{N^2}{n} \frac{N - n}{N - 1} \sigma_X^2 - 2 N^2 \rho_{XY} & < 0 \\
 \frac{N^2}{n} \frac{N - n}{N - 1} \sigma_X^2 - 2 \frac{N^2}{n} \frac{N - n}{N - 1} \sigma_X^2 & < 0 \\
 - \frac{N^2}{n} \frac{N - n}{N - 1} \sigma_X^2  & < 0 \\
\end{split}
\]

To je očitno res, saj je $\sigma_X^2 > 0$ in zato $var(\hat{\tau}_Y ^ {(1)}) < var(\hat{\tau}_Y ^ {(0)}) $ velja vedno.
Zgoraj smo upoštevali še
\[
\begin{split}
cov(\bar{X}, \bar{Y}) & = cov(\bar{X}, \bar{Y} - \bar{X} + \bar{X}) \\
                                   & = cov(\bar{X}, \bar{Y} - \bar{X}) + cov(\bar{X}, \bar{X}) \\
                                   & = 0 + var(\bar{X}) \\
                                   & = \frac{\sigma_X^2}{n} \frac{N - n}{N - 1}
\end{split}
\]



% LITERATURA

\begin{thebibliography}{9}
\bibitem{veriznica}
	E. Zakrajšek, \emph{Verižnica} (1999) od 6 do 10.
	Dostopno na spletni učilnici FMF 2019/2020 predmeta Matematično modeliranje [22. 7. 2020]
\bibitem{zapiski}
	Zapiski s predavanj predmeta Matematično modeliranje.
\end{thebibliography}

\end{document}


